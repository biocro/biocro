% Here I'm trying to implement the thesis template for latex

\documentclass[10pt]{article}

%\usepackage[pdftex]{graphicx}
%\usepackage{textcomp}
\usepackage{amssymb,amsmath,longtable}
%\usepackage{hyperref}
%\usepackage{pdfpages}
%\usepackage{caption}
\usepackage[disable, backgroundcolor=white, textsize = small]{todonotes}
\usepackage{marginnote}
%\renewcommand{\marginnote}[2][]{}
\usepackage[top=2.5cm, bottom=2.5cm, left=0.5cm, right=5cm, heightrounded,
  marginparwidth=4.6cm, marginparsep=3mm]{geometry}
% \usepackage{draftwatermark}
% \SetWatermarkLightness{0.95}
% \SetWatermarkScale{4}
\begin{document}

\section*{BioCro Equations}

\subsection*{Canopy Radiation}

\begin{align}
 \delta  &=  -23.5 \cdot \cos\biggl(\frac{360(D_{j}+10)}{365}\biggr) \hspace{2.5in} \label{eqn:delt} \\
 \cos(\theta)   &=  \sin(\Omega)\sin(\delta)+\cos(\Omega)\cos(\delta)\cos(15\cdot(t-t_{sn}))  \label{eqn:costheta}\\
 I_{dir}   &=  I_{s}\alpha^{\frac{(P/P_{o})}{\cos(\theta)}} \label{eqn:Idir}\\
 I_{\mathit{diff}}  &=   0.5 \cdot I_{s}\cdot(1-\alpha^{(P/P_{o})/\cos(\theta)})\cos(\theta) \\%\label{eqn:Idiff}\marginnote{P is defined as a constant (?)}\\
\tfrac{1}{2} \cos((15 \cdot t_{len})  &=  -\tan(\Omega)\tan(\delta) \label{eqn:costlen_old}  \\
t_\text{len} &= \frac{2cos^{-1}(-\tan(\Omega)\tan(\delta))}{15}\label{eqn:costlen}\\
t_\text{down} &= 12-t_\text{len}/2\label{eqn:tdown}\\
t_\text{up} &= 12+t_\text{len}/2\label{eqn:tup}
\end{align}
\todo[inline]{what is the relevance of equation \ref{eqn:costlen_old}? Steve H's thesis contains the original, equation \ref{eqn:costlen}}


\subsection*{Weather Downscaling}
\begin{align}
T_\text{mean} &= \frac{1}{2}\left(T_\text{max} + T_\text{min}\right)\label{eqn:Tmean}\\
T_\text{range} &= T_\text{max} - T_\text{min}\label{eqn:Trange}\\
T_\text{excursion}  &=  \sin\biggl(2 \pi \frac{h_{r}-10}{24}\biggr) \label{eqn:excur} \\
T_\text{air}  &=   T_\text{mean} + T_\text{range} \cdot T_\text{excursion} \label{eqn:Tair} 
\end{align}

\subsection*{Canopy Radiation}
\todo[inline]{combine or use clearly distinguished titles for different sections on canopy radiation; energy balance, etc} 
\begin{align}
 q  &=  \frac{n_r}{n} \label{eqn:q} \\
 N_\text{eff}  &=  \frac{\frac{(1-q)}{q}}{C_{ov}^{2}} \label{eqn:Neff}\\
 r^{\sim}  &=  \frac{m_r}{n} \label{eqn:rsim} \\
 h  &=  \frac{r^{\sim}}{q} \label{eqn:h} 
\end{align}


 \subsection*{C4 Photosynthesis}
 From Collatz 1992 Coupled Photosynthesis-Stomata1 Conductance Model for Leaves of C4 Plants. Aust. J. Plant Physiol. 19 519-538
 \begin{align}
 V_\text{max}&=\frac{V_{\text{max}_0}Q_{10}^{\frac{T_\text{leaf}-25}{10}}}{\left(1+\exp(0.3(T_\text{lower}-T_\text{leaf})\right)\left(1+\exp(0.3(T_\text{leaf}-T_\text{upper})\right)}\label{eqn:Vmax}\\
 R_d&= \frac{R_0Q_{10}^{\frac{T_\text{leaf}-25}{10}}}{1+\exp(1.3(T_\text{leaf}-55))}\label{eqn:Rd}\\
 k_t&=kQ_{10}^{\frac{T_\text{leaf}-25}{10}}\label{eqn:kt}\\
 c_i &=c_a-\frac{1.6A_nP}{g_s}\ref{eqn:ci}\\
 A_\text{net}&=A_\text{gross}-R_d\label{eqn:Anet}\\
 M &= \min\left[\frac{\scriptstyle(V_\text{max}+\alpha_{\text{slope}} I_\text{abs}) \pm \sqrt{(V_\text{max}+\alpha_\text{slope}I_\text{abs})^{2} - 4  (V_\text{max} \alpha_{\text{slope}}  I_\text{abs}) \theta_\text{curve}}}{2  \theta_\text{curve}}\right] \label{eqn:M} \\
 A_\text{gross} &= \min\left[ \frac{ \left( M + k_{t} \cdot \frac{c_{i}}{P} \right) \pm \sqrt{ \left(M + k_{t} \cdot \frac{c_{i}}{P} \right)^2 - \left( 4 \cdot M \cdot k_{t} \cdot \frac{c_{i}}{P} \cdot \beta \right)}}{2 \cdot \beta}\right] \label{eqn:Agross}
 \end{align}

\subsection*{Effect of Specific Leaf Nitrogen on C4 photosynthesis}
\begin{align}
V_\text{max}&= m_\text{vmax} N_\text{leaf} + c_\text{vmax}\\
R_\text{d}&= m_\text{Rd} N_\text{leaf} + c_\text{Rd}\\
\alpha_\text{slope}&= m_\alpha  N_\text{leaf} + c_\alpha
\end{align}


\newpage
\subsection*{C3 Photosynthesis}

From Appendix 2 in Bernacchi et al 2003 Plant, Cell and Environment 26, 1419–1430  doi: 10.1046/j.0016-8025.2003.01050.x:

\begin{align}
A &= \left(1-\Gamma^\ast/c_i\right) \label{eqn:A}\\
w_c &= \frac{V_{c\text{max}}c_i}{c_i+K_c(1+O_a/K_0)}\label{eqn:wc}\\
w_j &= \frac{Jc_i}{4.5c_i + 10.5\Gamma^{*}}\label{eqn:wj}\\
\Gamma^\ast &= exp(19.02 - 37.83/(R(T_\text{leaf}+273.15)))\label{eqn:gammaast}\\
K_c &= exp(38.05-36.38/R(T_\text{leaf}+273.15))\label{eqn:Kc}\\
K_0&=exp(20.30-36.38/R(T_\text{leaf}+273.15))\label{eqn:K0}\\
V_{c,\text{max}}&=V_{c,\text{max}@25C}\cdot \exp(26.35-65.33/R(T_\text{leaf}+273.15))\label{eqn:Vcmax}\\
J&=\frac{Q_2+J_\text{max,T}-\sqrt{(Q_2+J_\text{max,T})^2-4\Theta_{PSII}Q_2J_{\text{max},T}}}{2\Theta_{\text{PSII}}}\label{eqn:J}\\
J_{\text{max},T}&=J_{\text{max}@25C}\exp(17.57-43.54/(R(T_\text{leaf}+273.15))) \label{eqn:Jmaxt}\\
\Theta_\text{PSII} &= 0.76+0.018T_\text{leaf}-3.7\cdot10^{-4}T_\text{leaf}^2\label{eqn:ThetaPSII}\\
Q_2&=Q\cdot k\cdot \Phi_\text{PSII,max}\cdot\beta_\Phi\label{eqn:Q2}\\
\Phi_\text{PSII,max}&= 0.352 + 0.022T_\text{leaf} - 3.4\cdot10^{-4} T_\text{leaf}^2\label{eqn:PhiPSIImax}
\end{align}

From Appendix 1, Equations 7-9 in Long 1991 Plant, Cell and Environment 14, 729-739. doi:10.1111/j.1365-3040.1991.tb01439.x:

\begin{align}
% tmp = (1.673998 - 0.0612936 * LeafT + 0.00116875 * pow(LeafT,2) - 8.874081e-06 * pow(LeafT,3)) / 0.735465;
c_i &= 0.7c_a\left(\frac{1.6740-6.1294\cdot10^{-2}T_\text{leaf}+1.1688\cdot10^{-3}T_\text{leaf}^2-8.8741\cdot10^{-6}T_\text{leaf}^3}{0.73547}\right)\label{eqn:ci}\\
c_i&=0.7c_a @ 25^{\circ}C \ref{eqn:ci}\\
%% c3photo.c: tmp = (0.047 - 0.0013087 * LeafT + 2.5603e-05 * pow(LeafT,2) - 2.1441e-07 * pow(LeafT,3)) / 0.026934;
O_i&=210\left(\frac{4.7000\cdot10^{-2}-1.3087\cdot10^{-3}T_\text{leaf}+2.5603\cdot10^{-5}T_\text{leaf}^2-2.1441\cdot10^{-7}T_\text{leaf}^3}{2.6934\cdot10^{-2}}\right)\label{eqn:Oi}\\
O_i&=O_a  @ 25^{\circ}C\ref{eqn:Oi}\\
\phi&=\frac{A_{\text{I}=50}-A_{\text{I}=25}}{25f}\label{eqn:phi}
%LCP=R_d/\phi\\
\end{align}
\todo[inline]{is there a reason not to divide by the denominator when it is constant?}

\subsection*{Water Stress}
\begin{align}
h_{s} &= \frac{e_{l} - \rho_{va}}{e_{l}} \cdot 100 \label{eqn:hs} \\
g_{s} &= g_{0} + g_{1} \cdot A_{\text{gross}} \cdot \frac{h_{s}}{c_{a}} \label{eqn:gs} \\
% g_{w,\text{mod}} &= \left( \frac{\Psi_{l} - \Psi_{t}}{1000} \right) \cdot g_{ws} \label{eqn:gwmod} \\
% g_1 &= g_1 \cdot (1-g_{w,\text{mod}}) \label{eqn:g1} \\
\text{Four options for water stress model:} \label{eqn:gws}\\
g_\text{ws, linear}&= \frac{W_s-W_p}{F_c-W_p}\\
g_\text{ws, logistic}&= \frac{1}{1+\exp\left(\frac{\frac{1}{2}(F_c+W_p)-W_s)}{\phi_i}\right)}\\
g_\text{ws, exponential}&= \frac{1-\exp\left(\frac{F_c-W_s}{F_c-W_p}+\frac{W_p}{1-W_p}\right)}{0.631206}\\
g_\text{ws, none}&= 1\\
\text{Calculate $g_s$ and $A_n$ under water stress:}\\
g_{s}^{\text{water stress}}&=g_\text{ws,*}g_s\\
A_{n}^{\text{water stress}}&=g_\text{ws,*}A_n
\end{align}
\todo[inline]{should there be only one equation for Anet(?) is either correct? The first seems strange in that it implies water limited Anet equals Anet times humidity}

\subsection*{Canopy Energy Balance}

\begin{align}
 J_a &= 2 \cdot I_{\text{abs}} \cdot \left(\frac{1-r-\tau}{1-\tau}\right) \cdot \ell \label{eqn:Ja} \\
 L_b &= (2.126 \cdot 10^{-5} + 1.48 \cdot 10^{-7} \cdot T_\text{air}) / 0.004 \cdot \sqrt{L_w / u_{\text{layer}}} \label{eqn:Lb}\\
 u_a &= \frac{u \cdot 0.41}{log((u-d)/z_o)} \hspace{3.1in} \label{eqn:ua} \marginnote{I can not reconcile units}\\
 g_a &= \frac{(u_{a}^{2}/u_{\text{layer}}) \cdot L_b}{(u_{a}^{2}/u_{\text{layer}}) + L_b} \label{eqn:ga} \marginnote{I can not reconcile units}\\
 \rho_{v}^{\prime} &= 610.78 \cdot \exp(\left( 17.269 \cdot \frac{T_a}{T_a + 237.3} \right)) \label{eqn:rhop} \\
 \Delta \rho_{va} &= \rho_{v}^{\prime} \cdot \left( 1-\frac{h_s}{100} \right) \label{eqn:Deltarho} \\
 \gamma &= \frac{\rho \cdot c_p}{\lambda} \label{eqn:gamma} \\
 s &= 18 \cdot (2501 - 2.373 \cdot T_a) \cdot \left(\frac{\rho_{v}^{\prime}}{8.314 \cdot (T_a + 273)^2} \right) \label{eqn:s} \\
 R_{lc} &= 4 \sigma \cdot (273+T_{\text{air}})^3 \cdot \Delta T \label{eqn:Rlc} \\
 \Phi_N &= J_a - R_{lc} \label{eqn:PhiN} \\
 \Delta T &= T_{\text{leaf}} - T_{\text{air}} = \frac{\Phi_n \left(\frac{1}{g_a}+\frac{1}{g_c} \right)}{\lambda \biggl[ s+ \gamma \left( 1 + \frac{g_a}{g_c} \right) \biggr]} - \frac{\lambda \Delta \rho_{va}}{\lambda \biggl[ s+\gamma\left(1+\frac{g_a}{g_c}\right) \biggr] } \label{eqn:DeltaT} \marginnote{should thermal conductivity be in this equation?} \\
 E &= \frac{ s \cdot \Phi_N + \lambda \cdot g_a \cdot \Delta \rho_{va}}{\lambda \cdot [ s + \lambda \cdot (1+g_a / g_c)]} \label{eqn:E} \\
 \mathbf{E_c} &= \sum_{\text{layer}=1}^\text{N} (\mathbf{E_{\text{sun}}} \cdot l_{\text{sun}}) + (\mathbf{E_{\text{shade}}} \cdot l_{\text{shade}}) \label{eqn:Ec}\\ 
 \mathbf{E_{\text{tot}}} &= \sum_{\text{day}=1}^{365}\sum_{\text{hr}=1}^{24}  \mathbf{E_c} \label{eqn:Etot}
\end{align}

\subsection*{Sun / Shade Canopy}
\begin{align}
 k &= \frac{\sqrt{\chi^2 + \tan^2(\theta)}}{\chi + 1.744 \cdot[\chi + 1.183]^{-0.733}} \label{eqn:k} \\
 F_\text{sun} &=  \frac{1 - \exp[-k\cdot F_\text{canopy}]}{k} \label{eqn:Fsun} \\
 F_\text{shade} &= F_{\text{canopy}} - F_{\text{sun}} \label{eqn:Fshade} \\
 I_\text{sun} &= k\cdot I_{beam} + I_\text{diff} + I_\text{scat} \label{eqn:Isun} \\
I_\text{beam} &= I_\text{dir}\cos(\theta)\\
 I_\text{shade} &= I_{\text{diff}} + I_\text{scat} \label{eqn:Ishade} \\
I_\text{diff} & = I_\text{od} \exp(-k_dF_\text{canopy})\\
 I_{\text{scat}} &= I_\text{beam} \exp(-k \sqrt{\alpha_\text{scat}}F_\text{canopy})- I_\text{beam}\exp(-kF_\text{canopy}) \label{eqn:Iscat}\\
\end{align}

\subsection*{Total Canopy Assimilation}
\begin{align}
 A_c &= (A_{c,\text{sun}}\cdot F_{\text{sun}}) + (A_{c,\text{shade}} \cdot F_{\text{shade}}) \label{eqn:Ac} \\
 F_{\text{sun}} &= \sum_{\text{layer}=1}^\text{N} l_{\text{sun}} ; \;  l_{\text{sun}} = \frac{1-e^{(-k\cdot F_{\text{sun}})}}{k} \label{eqn:Fsun2} \\
 F_{\text{shade}} &= \sum_{\text{layer}=1}^\text{N} \ell_{\text{shade}}; \; \ell_{\text{shade}}=F_{\text{sun}}-\ell_{\text{sun}} \label{eqn:Fshade2} \\
 F_{\text{canopy}} &= F_{\text{sun}} + F_{\text{shade}} \hspace{3.1in} \label{eqn:Fcanopy2}\\
 A_c &= \sum_{\text{layer}=1}^\text{N} (A_{c,\text{sun}} \cdot F_{\text{sun}}) + (A_{c,\text{shade}} \cdot F_{\text{shade}}) \label{eqn:Ac2} \\
 A_{c,\text{tot}} &= \sum_{\text{day}=1}^{365}\sum_{\text{hr}=1}^{24} A_c \label{eqn:Actot} \\
 g_c &= \sum_{\text{layer}=1}^\text{N}(g_{s,\text{sun}}\cdot l_{\text{sun}}) + (g_{s,\text{shade}}\cdot l_{\text{shade}}) \label{eqn:gc}\\ 
 g_{c,\text{tot}} &= \sum_{\text{day}=1}^{365}\sum_{\text{hr}=1}^{24}g_c \label{eqn:gctot} 
\end{align}
\todo[inline]{is $\ell_\text{sun}\equiv l_\text{sun}$?}

\subsection*{Respiration}
\begin{align}
 % from Miguez 2009: R_{\text{total}} &= (a \cdot A_{\text{gross}}) + (b_{\text{leaf}} \cdot \omega_{\text{leaf}}) + (b_{\text{stem}} \cdot \omega_{\text{stem}}) + (b_{\text{root}} \cdot \omega_{\text{root}}) \label{eqn:Rtotal} \\
  R_{\text{total}} &= a A_n + b_{\text{stem}}\Delta\omega_{\text{stem}} + b_{\text{root}}\Delta \omega_{\text{root}} + b_\text{storage}\Delta \omega_\text{storage} \label{eqn:Rtotal} 
\end{align}


\subsection*{Allocation}

\begin{align}
% Miguez 2009 A_{\text{stroot}} &= |\omega_{\text{stroot}} \cdot k_{\text{stroot}}|  ; k_{\text{storage}}<0 \label{eqn:Astroot} \\
%Willow:
A_{\text{storage}} &= |\min(0, \omega_{\text{storage}} \cdot k_{\text{storage}})|  \label{eqn:Astroot} \\
% Miguez 2009 A_{\text{total}} &= A_c + A_{\text{seed}} + A_{\text{stroot}} \label{eqn:Atotal} \\
%Willow:
A_{\text{total}} &= A_{\text{leaf}} + A_{\text{stem}} + A_{\text{root}} + A_{\text{storage}}  \label{eqn:Atotal} \\
\omega_{\text{leaf}} &= \omega_{\text{leaf}} + (A_{\text{total}} \cdot k_{\text{leaf}}) \label{eqn:omegaleaf} \\
\omega_{\text{stem}} &= \omega_{\text{stem}} + (A_{\text{total}} \cdot k_{\text{stem}}) \label{eqn:omegastem} \\
%Miguez 2009 \omega_{\text{sroot}} &= \omega_{\text{sroot}} + (A_{\text{total}} \cdot k_{\text{sroot}}) \label{eqn:omegasroot} \\
%Miguez 2009 \omega_{\text{stroot}} &= \omega_{\text{stroot}} + (A_{\text{total}} \cdot k_{\text{stroot}})  \\
 \omega_{\text{stroot}} &= \omega_{\text{storage}} + (A_{\text{total}} \cdot k_{\text{storage}})  \\
%Miguez 2009 \omega_{\text{froot}} &= \omega_{\text{froot}} + (A_{\text{total}} \cdot k_{\text{froot}}) \label{eqn:omegafroot} \\
 \omega_{\text{root}} &= \omega_{\text{root}} + (A_{\text{total}} \cdot k_{\text{root}}) \label{eqn:omegaroot} \\
 \Psi_{\text{adl}} &< \Psi_{\text{pt}}  \\
 k_{\text{leaf}} &= k_{\text{leaf}} \cdot k_{\text{mod}} \\
 k_{\text{stem}} &= k_{\text{stem}} \cdot k_{\text{mod}} \\ 
% k_{\text{stroot}} &= k_{\text{stroot}} \cdot k_{\text{mod}} \\ 
 k_{\text{storage}} &= k_{\text{storage}} \cdot k_{\text{mod}}  \\ 
 k_{\text{mod}} &= (\Psi_{\text{adl}} - \Psi_{\text{pt}}) \cdot \Psi_g ; 0 \leq k_{\text{mod}} \leq 1 \label{eqn:kmod} \\
 \Delta F_{\text{canopy}} &= \frac{\omega_{\text{leaf}}}{Sp_{\text{leaf}}} \label{eqn:DeltaFcanopy} \\
 \Delta L_{\text{stem}} &= \frac{\omega_{\text{stem}}}{Sp_{\text{stem}}} \label{eqn:DeltaLstem} \\
% Miguez 2009 \Delta L_{\text{sroot}} &= \frac{\omega_{\text{root}}}{Sp_{\text{sroot}}} \label{eqn:DeltaLsroot} \\
 \Delta L_{\text{sroot}} &= \frac{\omega_{\text{sroot}}}{Sp_{\text{sroot}}} \label{eqn:DeltaLsroot} \\
 \Delta L_{\text{storage}} &= \frac{\omega_{\text{storage}}}{Sp_{\text{storage}}}\\
% New for coppicing trees:
\text{Stem}_{\text{coppice}}&=0.95 - \omega_{\text{stem}}
\end{align}

\todo[inline]{should restrictions on values of $k$ in equations \ref{eqn:Astroot} and \ref{eqn:kmod} be moved to the parameter definitions?}
\todo[inline]{would it make sense to subscript values of $\omega$ with $t$, $t+1$ when updating them to avoid confusion?}
\todo[inline]{$\Delta t$ is the timescale for updating biomass.; need to define $\Delta T$ for both daily and hourly}

\subsection*{Soil Evaporation}

\begin{align}
 E_{\text{soil}} &= \sum \frac{(\Psi_{\text{si}} - g \cdot z_i - \Psi_x)}{R_{\text{si}}+R_{\text{ri}}} \label{eqn:Esoil} \\
 R_{\text{ri}} &= R_{r} \cdot \frac{\sum L_i}{L_i} \label{eqn:Rri} \\
 \Psi_x &= \sum \frac{(\Psi_{\text{si}}-q_w \cdot z_i)}{R_{\text{si}}+R_{\text{ri}}} / \sum \frac{1}{R_{\text{si}}+R_{\text{ri}}} \label{eqn:Psix} \\
 \Psi_L &= \Psi_x  - E \cdot R_L \hspace{3.2in} \label{eqn:PsiL}\\
 E_d &= \left\{ \begin{array}{ll}
        E_p , & \theta^{*} \geq \theta_1 \\
        E_p\biggl( \frac{\theta-\theta_2}{\theta_1 - \theta_2} \biggr), & \theta_2<\theta^*<\theta_1 \\
         0, & \theta^{*} \leq \theta_2 
                \end{array} \right. \label{eqn:Ed}\\
 \theta_{i+1} &= \theta_i - \frac{E_i \cdot \theta_i}{\rho_w \cdot d_s} \label{eqn:theta}\\
 g_{a,\text{soil}} &= \frac{(2.126 \cdot 10^{-5})+(1.48 \cdot 10^{-7}) \cdot T_{\text{soil}}}{\left(0.004 \cdot \sqrt{\frac{S_{size}}{u_{\text{soil}}}} \right)} \label{eqn:gasoil} \\
 R_{lc,\text{soil}} &= ((4 \sigma)\cdot(273 + T_{\text{soil}})^3 \cdot \Delta T) \label{eqn:Rlcsoil} \\
 J_{a,\text{soil}} &= 2 \cdot I_{\text{soil}} \cdot \biggl(\frac{1-S_r-S_{\tau}}{1-S_{\tau}}\biggr) \label{eqn:Jasoil} \\
 \Phi_{N,soil} &= J_{a,soil} - R_{lc,soil} \label{eqn:PhiNsoil} \\
 E_\text{soil} &= \frac{s \cdot \Phi_{N,soil} + \lambda \cdot g_{a,soil} \cdot \Delta \rho_{va}}{\lambda \cdot [s+\gamma]} \label{eqn:E2} \marginnote{is ``soil'' subscript correct?}
\end{align}



\subsection*{Soil Energy Balance}
\begin{align}
 HS_{\text{soil}} &= HO_{\text{soil}} \cdot exp\biggl[\frac{h_{\text{soil}}}{46.97 \cdot (T_\text{soil} + 273.16)}\biggr] \label{eqn:HSsoil} \\
 HO_{\text{soil}} &= 1.323 \cdot exp\biggl[\frac{17.27 \cdot T_\text{soil}}{273.3 + T_\text{soil}}\biggr]\Bigg/ T_\text{soil} + 273.16 \label{eqn:HOsoil} \\
 G_{\text{soil}} &= -\lambda_{\text{soil}} \frac{\delta T}{\delta x} \label{eqn:Gsoil} \\
 G_{\text{soil}} &= -\lambda_{\text{soil}} \cdot \biggl[\frac{T_2 - T_\text{soil}}{\Delta z}\biggr] + (T_\text{soil} - T_l) \cdot C \cdot \frac{\Delta z}{(2 \cdot \Delta t)} \label{eqn:Gsoil2} 
\end{align}
\todo[inline]{should denominator in equation \ref{eqn:Gsoil} be $\delta z$?}

\todo[inline]{what is $t_\text{l}$?}
\todo[inline]{$C$ in from equation \ref{eqn:Gsoil2} is undefined - is this the specific heat of soil?}
\newpage

\section*{Definition of Terms } 

\begin{center}
\small
\begin{longtable}{l l p{3in} p{0.5in}}
%\caption{\textbf{Variables in the model}} \\
\hline \textbf{Term} & \textbf{Units} & \textbf{Definition} & \textbf{Value} \\ \hline 

\endfirsthead

\multicolumn{4}{c}%
{{\bfseries \tablename\ \thetable{} -- continued from previous page}} \\
\hline \textbf{Term} &
\textbf{Units} &
\textbf{Definition} &
\textbf{Value}  \\ \hline 
\endhead

\hline 
\endlastfoot

$A_{\text{gross}}$\ref{eqn:Agross}	&	$\mu mol\, mol^{-1}$ &	Gross rate of CO$_2$ uptake per unit leaf area	&	-	\label{parm:Agross}\\
$A_{\text{net}}$\ref{eqn:Anet}	&	$\mu mol\, mol^{-1}$	&	Net rate of CO$_2$ uptake per unit leaf area	&	-	\label{parm:Anet}\\
$A_{net,\text{water stress}}$ &	$\mu mol\, mol^{-1}$ & $A_\text{net}$ under water stress & \\
$A_c$	&	$\mu mol\, mol^{-1}$	&	Net canopy rate of CO$_2$ uptake per unit ground area	&	-	\\
$A_{c,\text{tot}}$	&	$g\, m^{-2}\, yr^{-1}$ 	&	A$_c$ integrated over the course of a year	&	-	\\
$A_{c,\text{sun}}$	&	$mol\, mol^{-1}$	&	Net rate of CO$_2$ uptake per unit area sunlit leaves	&	-	\\
$A_{c,\text{shade}}$	&	$mol\, m^{-2}\, s^{-1}$ 	&	Net rate of CO$_2$ uptake per unit area shaded leaves	&	-	\\
$A$\ref{eqn:A}	&	$\mu mol\, mol^{-1}$	&	Predicted rate of CO$_2$ uptake	&	-	\label{parm:A}\\
$c_a$	&	$\mu mol\, mol^{-1}$	&	Atmospheric CO$_2$ concentration	&	378	\\
$c_i$\ref{eqn:ci}\label{parm:ci}\marginnote{undefined}\\
$C$ & J$^\circ$C$^{-1}$m$^-3$& volumetric heat capacity  & \\
$a$	&	Dimensionless 	&	Coefficient for growth respiration	&	0.2	\\
$\alpha$	&	dimensionless	&	Atmospheric transmittance	&	0.85	\\
$\alpha_{\text{slope}}$	&	mol mol$^{-1}$ 	&	The quantum yield of CO$_2$ uptake determined by the initial slope of the response of A versus I$_{abs}$	&	0.04	\\
$b_{\text{leaf}}$	&	Dimensionless	&	Coefficient for maintenance respiration for leaf	&	0.03	\\
$b_{\text{stem}}$	&	Dimensionless	&	Coefficient for maintenance respiration for stem	&	0.015	\\
$b_{\text{root}}$	&	Dimensionless	&	Coefficient for maintenance respiration for root	&	0.01	\\
$\beta$	&		&	C$_4$ curvature parameter	&	0.93	\\
$\beta_\Phi$ & \% & Fraction of absorbed quanta reaching PSII & \\
$c_\alpha $ & $ mol\, mol^{-1}$& intercept of linear relationship between quantum yield of leaf photosynthesis rate and specific leaf nitrogen & \\
$c_i$	&	$\mu mol\, mol^{-1}$	&	Intercellular concentration of O$_2$ in air corrected for solubility relative to 25$^\circ$C	&	\\
$c_p$	&	$J\, kg^{-1}\, K{-1}$	&	Specific heat capacity of dry air 	&	1010	\\
$C_{ov}$	&	Dimensionless	&	Coefficient of Variation for probability of rain in each month	&	-	\\
$c_\text{Rd} $ & $\mu mol\, m^{-2} s^{-1}$& intercept of linear relationship between leaf dark respiration rate and specific leaf nitrogen & \\
$c_\text{vmax} $ & $\mu mol\, m^{-2} s^{-1}$& intercept of linear relationship between maximum rate of carboxylation and specific leaf nitrogen & \\
$d_s$	&	m	&	Soil depth	&	-	\\
$D_j$	&	d	&	day of year	&	-	\\
$D_{\text{start}}$	&	d	&	Day of year on which the sinusoidal temperature function is assumed to start	&	45	\\
$d$	&	dimensionless	&	Zero plane displacement	&	0.77	\\
$\delta$\ref{eqn:delt}	&	degrees	&	Solar declination	&	-	\label{parm:delt}\\
$e_l$	&	kPa	&	Saturated water VPD in the leaf	& \marginnote{is ``saturated VPD'' an oxymoron?}	-	\\
$E$	&	$J\, mol^{-1}$	&	Activation energy	&	$R_{d}=66405$	\\
	&		&		&	$V_{\text{max}}=6800$	\\
$E_d$ & & & \marginnote{undefined from equation \ref{eqn:Ed}} \\
$E_i$ & & & \marginnote{undefined from equation \ref{eqn:theta}} \\
$E_c$	&	$mmol\,m^{-2}\, s^{-1}$	&	Instantaneous canopy evapo/transpiration rate	&	-	\\
$E_d$	&	$g\, m^{-2}\, s^{-1}$	&	Potential soil evaporation	&	-	\\
$E_l$	&	$mmol\,m^{-2}\,s^{-1}$ 	&	Evapo/transpiration rate at sunlit/shaded leaves in a canopy layer	&	-	\\
$E_p$	&	$g\, m^{-2}\, s^{-1}$	&	Actual soil evaporation	&	-	\\
$E_{R_d}$ & J mol$^{-1}$ & Activation energy of R$_d$ & - \\
$E_\text{tot}$	&	$mmol\, m^{-2}\, yr^{-1}$	&	E$_c$ integrated over the course of a year	&	-	\\
$E_{V_\text{max}}$ & J mol$^{-1}$ & Activation energy of V$_\text{cmax}$& - \\
$f$ & & fraction of light not absorbed by photosynthesis & 0.23 \\
$f_{s,l}$ & & fraction of sunlit leaves at depth $l$ ($l$ is cumulative leaf area index from top) & \\
$F_c$ & $m^3m^{-3}$ & Field Capacity & \\
$F_{\text{canopy}}$	&	$m^2\, m^{-2}$	&Cumulative canopy leaf area index from top at depth	&	9	\\
$F_{\text{shade}}$	&	$m^2\, m^{-2}$	&	Canopy shaded leaf area index	&	-	\\
$F_{\text{sun}}$	&	$m^2\,m^{-2}$	&	Canopy sunlit leaf area index	&	-	\\
$F_{\text{sum}}$	&	$m^2\, m^{-2}$	&	Summed leaf area index from top of canopy to layer considered in calculation	&	-	\\
$G_{\text{soil}}$	&	$W\,m^{-2}$	&	Soil heat flux	&	-	\\
$g$	&	$m\, s^{-2}$	&	Gravitional constant	&	9.8	\\
$g_a$	&	$mmol\,m^{-2}\, s^{-1}$	&	Leaf boundary layer conductance	&	-	\\
$g_c$	&	$mmol\,m^{-2}\, s^{-1}$	&	Canopy conductance of CO$_2$ 	&	-	\\
$g_\text{c, root}$	&	& & \marginnote{undefined}\\
$g_s$	&	$mmol\,m^{-2}\, s^{-1}$	&	Leaf stomatal conductance	&	-	\\
$g_0$	&	dimensionless	&	Stomatal slope factor	&	3	\\
$g_1$	&	dimensionless	&	Stomatal intercept factor	&	0.08	\\
$g_{s,\text{sun}}$	&	$mmol\, m^{-2}\, s^{-1}$	&	The sum of stomatal conductance of sunlit leaves	&	-	\\
$g_{s,\text{shade}}$	&	$mmol\, m^{-2}\, s^{-1}$	&	The sum of stomatal conductance of shaded leaves	&	-	\\
$g_\text{ws}$ & dimensionless & Species-specific water stress sensitivity factor & \\
$g_\text{ws*}$ & dimensionless & water stress stomatal conductance factor; see equations \ref{eqn:gws} &\\
$\gamma$	&	Pa K$^{-1}$	&	psychrometer constant 	&	-	\\
$\Gamma^\ast$\ref{eqn:gammaast} & $\mu$mol mol$^{-1}$ & CO$_2$ compensation point in the absence of dark respiration & \label{parm:gammaast}\\
$h_r$	&	h	&	Hour of day	&	-	\\
$h_s$	&	\%	&	Relative humidity	&	-	\\
$h_{\text{canopy}}$	&	m	&	Height of canopy	&	5	\\
$h_{ms}$	&	m	&	Wind speed measurement height	&	2	\\
$h_{\text{layer}}$	&	m	&	Height of canopy layer above ground	&	-	\\
	I&	$\mu mol\, m^{-2}\, s^{-1}$ 	&	Photon flux	&	-	\\
$h$	&	$mm day^{-1}$	&	The amount of water received on a given rainy day	&	-	\\
$h_{\text{soil}}$	&	m	&	Water pressure head	&	-	\\
$HO_{\text{soil}}$	&	$kg\,m^{-3}$	&	Saturated humidity of the air at the soil surface	&	-	\\
$HS_{\text{soil}}$	&	$Kg\, m^{-3}$	&	Humidity of the air at the soil surface	&	-	\\
$I_{\text{abs}}$	&	$\mu mol\, m^{-2}\, s^{-1}$ 	&	Photon flux absorbed by either sunlit or shaded leaves within a canopy layer	&	-	\\
$I_{\text{dir}}$\ref{eqn:Idir}	&	$\mu mol\, m^{-2}\, s^{-1}$ 	&	Photon flux in direct solar beam	&	-	\label{parm:Idir}\\
$I_{\mathit{\text{diff}}}$	&	$\mu mol\,m^{-2}\, s^{-1}$ 	&	Photon flux in diffuse radiation	&	-	{eqn:Idiff}\\
$I_{\text{total}}$	&	$\mu mol\, m^{-2}\, s^{-1}$ 	&	Total photon flux incident on canopy	&	-	\\
$I_s$	&	$\mu mol\, m^{-2}\, s^{-1}$ 	&	Solar constant, photon flux in a plane perpedicular to the solar beam above the atmosphere	&	2600	\\
$I_d$ & & & \marginnote{undefined}\\
$I_{\ell,d}$ & & & \marginnote{undefined}\\
$I_{\text{short}}$	&	$\mu mol\, m^{-2}\, s^{-1}$ 	&	Short wave radiation component of incident light	&	-	\\
$I_\text{beam}$ & & flux density of beam radiation on horizontal surface at top of canopy & \\
$I_\text{od}$ & & flux density of diffuse radiation on horizontal surface at top of canopy & \\
$I_{\text{soil}}$	&	$\mu mol\, m^{-2}\, s^{-1}$ 	&	Solar radiation incident upon soil surface	&	-	\\
$I_\text{soil}$	&	$W\, m^{-2}$	&	Solar radiation on soil	&	-	\marginnote{which units for $I_\text{soil}$ are correct?}\\
$I_\text{sun}$	&	$\mu mol\, m^{-2}\, s^{-1}$ 	&	Mean I for leaves which receive direct solar radiation, i.e. are sunlit	&	-	\\
$I_\text{shade}$	&	$\mu mol\, m^{-2}\, s^{-1}$ 	&	Mean I for leaves shaded from direct solar radiation	&	-	\\
$I_{\text{scat}}$	&	$\mu mol\, m^{-2} s^{-1}$ 	&	Direct beam radiation scattered by surfaces within the canopy	&	-	\\
$J$\ref{eqn:J}\label{parm:J}\marginnote{undefined}\\
$J_{\text{max},T}$\ref{eqn:Jmaxt}\label{parm:Jmaxt}\marginnote{undefined}\\
$J_\text{a}$	&	$\mu mol\, m^{-2}\, s^{-1}$ 	&	Total solar radiation absorbed by either sunlit or shaded leaves within a canopy layer	&	-	\\
$J_\text{a, soil}$	&	$\mu mol\, m^{-2}\, s^{-1}$ 	&	Total solar radiation absorbed by soil	&	-	\\
$k$ &	dimensionless	&	Foliar absorption coefficient ($\alpha_\ell$ in Bernacchi 2003)	&	-	\\
$K_0$\ref{eqn:K0}\label{parm:K0}\marginnote{undefined}\\
$k_d$ & dimensionless & extinction coefficient for diffuse light & \\
$K_c$\ref{eqn:Kc}	&	$\mu mol\, mol^{-1}$ 	&	Michaelis Menton constant for the carboxylation of RuBISCO	&	460	\label{parm:Kc}\\
$K_{\text{CO}_2}$  	&	$mol\, m^{-2}\, s^{-1}$ 	&	Initial slope of photosynthetic CO$_2$ response	&	0.7	\\
$Kt$	&		&	C$_4$ slope factor	&	-	\\
$K_{o}$	&	$mmol\, mol^{-1}$	&	Michaelis Menton constant for oxygenation of RuBISCO 	&	330	\\
$k_{\text{slope}}$	&	Dimensionless	&	Initial slope of photosynthetic light response	&	0.04	\\
LN	&	$g\, m^{-2}$ 	&	Leaf nitrogen concentration	&	-	\\
$k_{\text{leaf}}$	&	Dimensionless	&	Partitioning coefficient for leaf	&	-	\\
$k_{\text{stem}}$	&	Dimensionless	&	Partitioning coefficient for stem	&	-	\\
$k_{\text{sroot}}$	&	Dimensionless	&	Partitioning coefficient for storage root	&	-	\\
$k_t$\ref{eqn:kt} & & temperature-dependent pseudo-first order rate constant with respect to $P_i$ (Collatz 1992) & \label{parm:kt}\\
$k_{\text{froot}}$	&	Dimensionless	&	Partitioning coefficient for fine root	&	-	\\
$k_{\text{stroot}}$	&	Dimensionless	&	Partitioning coefficient for structural root	&	-	\\
$\ell$ & & & \marginnote{undefined from Ja: equation \ref{eqn:Ja}}\\
$\ell_\text{sun}$ & & & \marginnote{undefined from equation \ref{eqn:Iells}}\\
$l_\text{sun}$ & & &\marginnote{undefined from equation \ref{eqn:Fsun2}}\\
$L_i$	&	$cm\, cm^{-3}$	&	Root density of ith zone	&	-	\\ 
$L_w$	&	m	&	Leaf width in the direction of the wind	&	0.04	\\
$\Delta L_\text{stem}$ & & & \marginnote{undefined}\\
$\Delta L_\text{sroot}$ & & & \marginnote{undefined}\\
$\lambda$	&	MJ/Kg	&	Latent heat of vapourisation	&	-	\\
$\lambda_\text{soil}$	&	$W/(m ^\circ C)$	&	Thermal conductivity for the soil surface	&	-	\\
$M$\ref{eqn:M} & & \marginnote{undefined}& \label{parm:M} -\\
$m_\alpha $ & $ mol\, mol^{-1} mmol N^{-1} m^{2}$& slope of linear relationship between quantum yield of leaf photosynthesis rate and specific leaf nitrogen & - \\
$m_\text{Rd} $ & $\mu mol\, m^{-2} s^{-1} mmol N^{-1} m^{2}$ & slope of linear relationship between leaf dark respiration rate and specific leaf nitrogen & - \\
$m_r$& mm month$^{-1}$& monthly precipitation rate& \\
$m_\text{vmax} $ & $\mu mol\, m^{-2} s^{-1} mmol N^{-1} m^{2}$& slope of linear relationship between maximum rate of carboxylation and specific leaf nitrogen & \\
$N_{\text{eff}}$& days/mo & effective length of rainy period & \marginnote{check units with equation \ref{eqn:Neff}}\\
$n$	&	day	&	The number of days in a month	&	29, 30, or 31	\\
$nr$	&	day 	&	The number of rainy days in a month	&	-	\\
$O_a$	&	$mmol\, mol^{-1}$	&	Atmospheric O$_2$ concentration	&	210	\marginnote{is this corected to 25C like O$_i$?}\\
$O_i$\ref{eqn:Oi}	&	$mmol\, m^{-2}\,s^{-1}$	&	Intercellular concentration of O$_2$ in air corrected for solubility relative to 25$^\circ$C	&	-	\label{parm:Oi}\\
$\omega_{\text{leaf}}$	&	g	&	Leaf biomass	&	-	\\
$\omega_{\text{stem}}$	&	g	&	Stem biomass	&	-	\\
$\omega_{\text{sroot}}$	&	g	&	Biomass of storage root	&	-	\\
$\omega_{\text{froot}}$	&	g	&	Biomass of fine root	&	-	\\
$\omega_{\text{stroot}}$	&	g	&	Biomass of structural root	&	-	\\
$\omega_{\text{storage}}$	&	g	&	Biomass of storage	&	-	\\
$\Omega$	&	degrees	&	Latitude	&	-	\\
$P$	&	kPa	&	Atmospheric pressure	&	\\
$P_o$	&	kPa	&	Standard atmospheric pressure at sea level	&	101.324	\\
$P_s$	&	kPa	&	Leaf surface partial pressure of CO$_2$ 	&	-	\\
$\Psi_\text{g}$& & & \marginnote{undefined}\\
$\Psi_l$	&	MPa	&	Leaf water potential	&	-	\\
$\Psi_\text{L}$& & & \marginnote{undefined}\\
$\Psi_t$	&	MPa	&	Threshold leaf water potential for decreasing gs	&	-	\\
$\Phi_\text{PSII,max}$\ref{eqn:PhiPSIImax}\label{parm:PhiPSIImax}\marginnote{undefined}\\
$\Phi_\text{N}$	&	W m$^{-2}$	&	Net radiation	&	-	\\
$\Phi_{N,soil}$	&	W m$^{-2}$	&	Net radiation at soil surface	&	-	\\
$\phi_i$ & & coefficient which controls spread of logistic function used to calculate water stress factor in \ref{eqn:gws}& \\
$\phi$\ref{eqn:phi}\label{parm:phi}\marginnote{undefined}\\
$\Psi_{\text{adl}}$	&	MPa	&	Average daily plant water potential	&	-	\\
$\Psi_{\text{pt}}$	&	MPa	&	Threshold water potential	&	-	\\
$\Psi_{\text{si}}$	&	MPa	&	Soil water potential of the ith layer	&	-	\\
$\Psi_x$	&	MPa	&	xylem water potential	&	-	\\
$q$	&	Dimensionless	&	The probability that there is no rainfall	&	-	\marginnote{during a month?}\\
$q_w$	&	kg s$^{-1}$	&	Flux of water	&	-	\\
$Q_2$\ref{eqn:Q2}\label{parm:Q2}\marginnote{undefined}\\
$Q_{10}$	&	dimensionless	&	Is the proportional rise in a parameter for a 10$^\circ$C increase in temperature	&	2	\\
$r$	&	dimensionless	&	Leaf reflection coefficient for total solar radiation	&	0.2	\\
$r^\sim$ & mm day$^{-1}$          & Mean daily rainfall in each month                            & - \\ 
$R$	&	$J\, k^{-1}\, mol^{-1}$ 	&	Real gas constant	&	8.314	\\
$R_L$	&	$m^3\, kg^{-1}\, s^{-1}$	&	Leaf resistance	&	-	\\
$R_{\text{si}}$	&	$m^3 kg^{-1} s^{-1}$	&	Soil resistance of the ith zone	&	-	\\
$R_{\text{ri}}$	&	$M^3 kg^{-1} s^{-1}$	&	root resistance of the ith zone	&	-	\\
$R_o$	&	$mol\, m^{-2}\, s^{-1}$ 	&	Dark respiration rate at $25^\circ C$	&	3	\\
$R_d$\ref{eqn:Rd}	&	$mol\, m^{-2}\, s^{-1}$ 	&	Dark respiration at a given temperature	&	-	\label{parm:Rd}\\
$R_{\text{lc}}$	&	$mol\, m^{-2}\, s^{-1}$ 	&	Longwave radiation	&	-	\\
$R_{lc,\text{soil}}$	&	$mol\, m^{-2}\, s^{-1}$	&	Soil longwave radiation	&	-	\\
$\rho_w$	&	$kg\, m^{-3}$	&	Density of water	&	1000	\\
$\rho_a$	&kPa& vapor pressure deficit in air&\marginnote{is this distinct from $\Delta \rho_\text{va}$?}\\
$\rho_a^\prime$	&&&\marginnote{undefined from equation \ref{eqn:Deltarho}}\\

$s$	&	kPa K$^{-1}$	&	Slope of saturated water vapor pressure change with respect to temperature (look up table)	&	-\marginnote{also defined by equation \ref{eqn:s}; is one correct?}	\\
$s_p$	&	dimensionless	&	Spectral imbalance	&	-	\\
$S_{\text{size}}$	&	m	&	Average size of soil particles	&	0.04	\\
$S_r$	&	Dimensionless	&	Soil reflectance	&	0.2	\\
$S_\tau$	&	Dimensionless 	&	Soil transmission	&	0.01	\\
$Sp_\text{leaf}$	&	gram m$^{-2}$ 	&	Specific leaf area	&	50	\\
$Sp_\text{stem}$	&	gram m$^{-1}$ 	&	Specific stem elongation factor	&	60	\\
$Sp_\text{froot}$	&	gram m$^{-1}$	&	Specific fine root elongation factor	&	10	\\
$Sp_\text{stroot}$	&	gram m$^{-1}$	&	Specific structural root elongation factor	&	60	\\
$\sigma$ & Wm$^{-2}$K$^{-4}$ & Stefan-Boltzmann constant & $5.67 \cdot 10^{-8}$\\
$t$	&	h	&	Time of day	&	-	\\
$t_\text{up}$ \ref{eqn:tup} & h & time of dawn &\label{parm:tup}\\
$t_\text{down}$\ref{eqn:tdown} & h & time of dusk & \label{parm:tdown}\\
$t_\text{len}$\ref{eqn:costlen}& h & day length & - \marginnote{is this a constant, 24?}\label{parm:costlen}\\
$t_\text{sn}$	&	h	&	Time of solar noon	&	12	\\
$T_\text{leaf}$	&	$^\circ$C	&	Leaf temperature	&	-	\\
$T_\text{air}$\ref{eqn:Tair}	&	$^\circ$C	&	Ambient air temperature	&	-	\label{parm:Tair}\\
$T_\text{mean}$\ref{eqn:Tmean} & 	$^\circ$C & Daily mean $T_\text{air}$ & \label{parm:Tmean}\\
$T_\text{range}$\ref{eqn:Trange} & 	$^\circ$C &  $\frac{T_\text{air}- T_\text{mean}}{T_\text{range}}$ & \label{parm:Trange}\\
$T_\text{excursion}$\ref{eqn:excur} & fraction & Difference between current $T_\text{mean}$   & \label{parm:excur}\\
$T_\text{soil}$	&	$^\circ$C	&	Soil surface temperature	&	-	\\
$T_\text{lower}$ &	$^\circ$C & Lower T limitation on photosynthesis& \\
$T_\text{upper}$ & 	$^\circ$C& Upper T limitation on photosynthesis  & \\

$T_1$	&	$^\circ$C	&	Annual mean air temperature	&	18	\\
$T_2$	&	$^\circ$C	&	Annual range in air temperature	&	2	\\
$T_3$	&	$^\circ$C	&	Average daily range in air temperature	&	7	\\
$T_4$	&	$^\circ$C	&	Maximum daily range in air temperature	&	7	\\
$\Delta T$ &	$^\circ$C& Temperature difference between (leaf or soil) and air &\\
$\tau$	&	Dimensionless 	&	Leaf transmittance coefficient	&		\\
$\Theta_{\text{curve}}$	&	dimensionless	&	Curvature parameter	&	-	\\
$\Theta^{*}$	&	$kg m^{-3}$	&	Actual volumetric water content	&	-	\\
$\Theta_1$	&	$kg\, m^{-3}$	&	The volumetric water content for maximizing Evaporation	&		\\
$\Theta_2$	&	$kg\, m^{-3}$	&	The volumetric water content for wilting point	&	-	\\
$\Theta_i$	&	$kg\, m^{-3}$	&	The volumetric water content of the ith day	&	-	\\
$\Theta_\text{PSII}$\ref{eqn:ThetaPSII}\label{parm:ThetaPSII}\marginnote{undefined}\\
$\Theta$	&	degrees	&	Solar zenith angle	&	-	\\
$u$	&	m s$^{-1}$	&	Measured wind speed at known height (2m)	&	2	\\
$u_{\text{layer}}$	&	m s$^{-1}$ 	&	Wind speed in a given canopy layer	&	-	\\
$u_{\text{soil}}$	&	m s$^{-1}$ 	&	Wind speed at soil surface	&	-	\\
$v$	&		&	Saturated water vapour concentration 	&	-	\\
$V_{\text{max}}$\ref{eqn:Vmax}	&	$mol\, m^{-2}\, s^{-1}$ 	&	Maximum rubP saturated rate of carboxylation at a given temperature	&	-	\label{parm:Vmax}\\
$V_{\text{max}_0}$	&	$mol\, m^{-2}\, s^{-1}$ 	&	Maximum rubP saturated rate of carboxylation at a given temperature	&	-	\\
$V_{c,\text{max}}$\ref{eqn:Vcmax}\label{parm:Vcmax}\marginnote{undefined}\\
$V_{c,\text{max}_0}$	&	$mol\, m^{-2}\, s^{-1}$ 	&	Maximum rubP saturated rate of carboxylation at $25^\circ C$	&	39	\\
VPD	&	kPa	&	Leaf-air water vapour pressure deficit	&	-	\\
$V_T$ & $mol\, m^{-2}\, s^{-1}$  &$V_\text{max}$ at current $T$ & \\
$w_c$\ref{eqn:wc} & 	$mol\, m^{-2}\, s^{-1}$\marginnote{units?} & RuBISCO limited rate of photosynthesis & \label{parm:wc}\\
$w_c$ & 	$mol\, m^{-2}\, s^{-1}$\marginnote{units?} & RuBP limited rate of photosynthesis & \\
$w_j$ \ref{eqn:wj}\label{parm:wj}\marginnote{undefined}\\
$W_p$ & $m^3m^{-3}$ & Wilting point & \\
$W_s$ & $m^3m^{-3}$ & Soil water content & \\
$z_o$	&	m	&	Roughness length	&	0.234	\\
$\chi$	&	dimensionless	&	The ratio of horizontal:vertical projected area of leaves in the canopy segment	&	1	\\
slope	&	$mol\, m^{-1}$ 	&	Initial slope of photosynthetic CO$_2$ response	&	0.7	\\
curve	&	dimensionless	&	Curvature parameter	&	0.83	\\
$Z$	&	m	&	Thickness of a soil layer	&	-	\\
\\
\end{longtable}
\end{center}

\end{document}
